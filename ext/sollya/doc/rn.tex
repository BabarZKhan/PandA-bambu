\subsection{RN}
\label{labrn}
\noindent Name: \textbf{RN}\\
constant representing rounding-to-nearest mode.\\
\noindent Description: \begin{itemize}

\item \textbf{RN} is used in command \textbf{round} to specify that the value must be rounded
   to the nearest representable floating-point number.
\end{itemize}
\noindent Example 1: 
\begin{center}\begin{minipage}{15cm}\begin{Verbatim}[frame=single]
> display=binary!;
> round(Pi,20,RN);
1.100100100001111111_2 * 2^(1)
\end{Verbatim}
\end{minipage}\end{center}
See also: \textbf{RD} (\ref{labrd}), \textbf{RU} (\ref{labru}), \textbf{RZ} (\ref{labrz}), \textbf{round} (\ref{labround}), \textbf{nearestint} (\ref{labnearestint})
