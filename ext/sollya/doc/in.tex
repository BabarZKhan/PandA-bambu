\subsection{in}
\label{labin}
\noindent Name: \textbf{in}\\
containment test operator\\
\noindent Usage: 
\begin{center}
\emph{expr} \textbf{in} \emph{range1} : (\textsf{constant}, \textsf{range}) $\rightarrow$ \textsf{boolean}\\
\emph{range1} \textbf{in} \emph{range2} : (\textsf{range}, \textsf{range}) $\rightarrow$ \textsf{boolean}\\
\end{center}
Parameters: 
\begin{itemize}
\item \emph{expr} represents a constant expression
\item \emph{range1} and \emph{range2} represent ranges (intervals)
\end{itemize}
\noindent Description: \begin{itemize}

\item When its first operand is a constant expression \emph{expr},
   the operator \textbf{in} evaluates to true iff the constant value
   of the expression \emph{expr} is contained in the interval \emph{range1}.

\item When both its operands are ranges (intervals), 
   the operator \textbf{in} evaluates to true iff all values
   in \emph{range1} are contained in the interval \emph{range2}.

\item \textbf{in} is also used as a keyword for loops over the different
   elements of a list.
\end{itemize}
\noindent Example 1: 
\begin{center}\begin{minipage}{15cm}\begin{Verbatim}[frame=single]
> 5 in [-4;7];
true
> 4 in [-1;1];
false
> 0 in sin([-17;17]);
true
\end{Verbatim}
\end{minipage}\end{center}
\noindent Example 2: 
\begin{center}\begin{minipage}{15cm}\begin{Verbatim}[frame=single]
> [5;7] in [2;8];
true
> [2;3] in [4;5];
false
> [2;3] in [2.5;5];
false
\end{Verbatim}
\end{minipage}\end{center}
\noindent Example 3: 
\begin{center}\begin{minipage}{15cm}\begin{Verbatim}[frame=single]
> for i in [|1,...,5|] do print(i);
1
2
3
4
5
\end{Verbatim}
\end{minipage}\end{center}
See also: \textbf{$==$} (\ref{labequal}), \textbf{!$=$} (\ref{labneq}), \textbf{$>=$} (\ref{labge}), \textbf{$>$} (\ref{labgt}), \textbf{$<=$} (\ref{lable}), \textbf{$<$} (\ref{lablt}), \textbf{!} (\ref{labnot}), \textbf{$\&\&$} (\ref{laband}), \textbf{$||$} (\ref{labor}), \textbf{prec} (\ref{labprec}), \textbf{print} (\ref{labprint})
