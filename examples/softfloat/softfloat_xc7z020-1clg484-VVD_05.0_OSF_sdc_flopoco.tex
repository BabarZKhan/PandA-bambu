%Generated at 2020-01-16 03:11:07.329944 with bambu  Version: PandA 0.9.6 - Revision 5dea469b31d6c7052279200a9bd3f69b10448cb6-PR-11 with arguments --no-iob --soft-float --hls-div --registered-inputs=top --panda-parameter=profile-top=1 --speculative-sdc-scheduling --device=TO_BE_OVERWRITTEN --evaluation --clock-period=5 --device=xc7z020-1clg484-VVD --flopoco
\begin{tabular}{|l|c|c|c|c|c|c|c|c|c|c|}
\hline
Benchmark Name                                   & Tot. Latency            & Num Cycles & LUTs     & Slices   & Registers & DSPs   & BRAMs & Clock Frequency & Clock Slack & HLS Time(s) \\
\hline
addition64:double\_prec\_addition\_0             & $ 1.315 \cdot 10^{-1} $ & $ 27     $ & $ 949  $ & $ 446  $ & $ 2011  $ & $ 0  $ & $ 0 $ & $ 205.38      $ & $ 0.13    $ & $ 0.51    $ \\
addition:single\_prec\_addition\_0               & $ 8.977 \cdot 10^{-2} $ & $ 19     $ & $ 459  $ & $ 183  $ & $ 721   $ & $ 0  $ & $ 0 $ & $ 211.64      $ & $ 0.28    $ & $ 0.45    $ \\
division64SRT4:double\_prec\_division\_0         & $ 2.041 \cdot 10^{-1} $ & $ 35     $ & $ 3486 $ & $ 1206 $ & $ 5152  $ & $ 0  $ & $ 0 $ & $ 171.47      $ & $ -0.83   $ & $ 0.49    $ \\
divisionSRT4:single\_prec\_division\_0           & $ 9.025 \cdot 10^{-2} $ & $ 19     $ & $ 856  $ & $ 320  $ & $ 1210  $ & $ 0  $ & $ 0 $ & $ 210.53      $ & $ 0.25    $ & $ 0.50    $ \\
multiplication64:double\_prec\_multiplication\_0 & $ 8.841 \cdot 10^{-2} $ & $ 15     $ & $ 898  $ & $ 374  $ & $ 1079  $ & $ 12 $ & $ 0 $ & $ 169.66      $ & $ -0.89   $ & $ 0.47    $ \\
multiplication:single\_prec\_multiplication\_0   & $ 3.894 \cdot 10^{-2} $ & $ 6      $ & $ 218  $ & $ 88   $ & $ 208   $ & $ 2  $ & $ 0 $ & $ 154.08      $ & $ -1.49   $ & $ 0.45    $ \\
subtraction64:double\_prec\_subtraction\_0       & $ 1.273 \cdot 10^{-1} $ & $ 27     $ & $ 953  $ & $ 487  $ & $ 2011  $ & $ 0  $ & $ 0 $ & $ 212.13      $ & $ 0.29    $ & $ 0.47    $ \\
subtraction:single\_prec\_subtraction\_0         & $ 8.632 \cdot 10^{-2} $ & $ 19     $ & $ 456  $ & $ 173  $ & $ 721   $ & $ 0  $ & $ 0 $ & $ 220.12      $ & $ 0.46    $ & $ 0.43    $ \\
\hline
Average                                          & $                     $ & $        $ & $      $ & $      $ & $       $ & $    $ & $   $ & $ 194.38      $ & $ -0.23   $ & $         $ \\
\hline
Overall                                          & $ 8.566 \cdot 10^{-1} $ & $ 167    $ & $ 8275 $ & $ 3277 $ & $ 13113 $ & $ 14 $ & $ 0 $ & $             $ & $         $ & $ 3.77    $ \\
\hline
\end{tabular}
%Benchmarks:
%src/fdiv.c --top-fname=single_prec_division --benchmark-name=divisionSRT4 --generate-tb=a=106\\\,b=57 --evaluation
%src/fmul.c --top-fname=single_prec_multiplication --benchmark-name=multiplication --generate-tb=a=106\\\,b=57 --evaluation
%src/fsub.c --top-fname=single_prec_subtraction --benchmark-name=subtraction --generate-tb=a=106\\\,b=57 --evaluation
%src/fsum.c --top-fname=single_prec_addition --benchmark-name=addition --generate-tb=a=106\\\,b=57 --evaluation
%src/f64div.c --top-fname=double_prec_division --benchmark-name=division64SRT4 --generate-tb=a=61\\\,b=3 --evaluation
%src/f64mul.c --top-fname=double_prec_multiplication --benchmark-name=multiplication64 --generate-tb=a=0\\\,b=0 --evaluation
%src/f64sub.c --top-fname=double_prec_subtraction --benchmark-name=subtraction64 --generate-tb=a=0\\\,b=0 --evaluation
%src/f64sum.c --top-fname=double_prec_addition --benchmark-name=addition64 --generate-tb=a=0\\\,b=0 --evaluation
